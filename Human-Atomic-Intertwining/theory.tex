
\documentclass[11pt]{article}
\usepackage[a4paper,margin=1in]{geometry}
\usepackage{amsmath, amssymb}
\usepackage{hyperref}
\usepackage{graphicx}
\usepackage{titlesec}
\usepackage{natbib}
\usepackage{enumitem}

\titleformat{\section}{\large\bfseries}{\thesection.}{1em}{}
\titleformat{\subsection}{\normalsize\bfseries}{\thesubsection.}{1em}{}

\title{\textbf{Brock P. Christoval's General Theory of Human Atomic Intertwining}}
\author{Brock P. Christoval \\ \textit{Independent Researcher}}
\date{}

\begin{document}
\maketitle

\begin{abstract}
This paper presents a novel hypothesis that reinterprets the phenomenon commonly labeled as ``coincidence''—unexpected re-encounters between individuals—as the product of subatomic entanglement rather than random chance. Introducing the concept of \textit{Broctytes}—hypothetical quantum particles encoding entanglement from human interactions—the theory proposes that these quantum links can influence movement and decision-making. At higher entanglement strengths, they subtly alter individual trajectories to facilitate renewed encounters, a process termed \textit{entanglement renewal}. This framework invites rigorous interdisciplinary study into quantum-level influences on human behavior and the structure of coincidence.
\end{abstract}

\section{Introduction: The Science of Coincidence and Synchronous Encounters}

Across cultures and histories, humans have described profound experiences of ``coincidence''—encounters with familiar individuals or events that seem improbably well-timed or deeply meaningful. These synchronous events, where physical proximity and emotional resonance align unexpectedly, are often dismissed as random chance. Yet, for many, such moments feel \textit{charged}—as though guided by an unseen force.

This theory, \textit{The General Theory of Human Atomic Intertwining}, seeks to explore a scientific basis for such phenomena. It proposes that some of these encounters are not mere coincidence, but manifestations of quantum-level connections between individuals—what we term \textit{Broctytes}. These connections may influence human behavior and movement patterns, subtly shaping life paths to bring individuals back into contact in space-time.

By introducing the concept of \textit{entanglement renewal}, this theory offers a new lens for understanding not only coincidence but also the broader category of synchronous experiences—those that appear orchestrated, symbolic, or emotionally significant. If proven, this hypothesis could revolutionize our understanding of human connectivity, volition, and even consciousness.

\section{Core Concepts}

\subsection{Broctyte}
A proposed quantum particle responsible for encoding subatomic-level entanglement between individuals. Each Broctyte’s strength depends on:
\begin{itemize}[nosep]
\item Interaction type (e.g., familial, romantic, transactional)
\item Interaction duration
\item Spatiotemporal alignment of individuals
\item Emotional and cognitive intensity
\end{itemize}

\subsection{Entanglement Renewal}
A dynamic mechanism where dormant Broctytes activate, subtly guiding motion and behavior to increase proximity between previously entangled individuals. These renewals are most likely when individuals are in motion and dissipate during prolonged stillness.

\section{Hypothesis (2025 Update)}

During interpersonal interaction, humans emit or establish Broctyte entanglements. These connections behave analogously to known quantum entanglement effects, potentially persisting and influencing trajectories. At sufficient strength, they function as attractors, increasing the likelihood of future re-encounters through what appears externally as coincidence.

\section{Addressing Probability as a Counter-Theory}

Coincidences are well-addressed in statistical literature. Key frameworks include:

\begin{itemize}[nosep]
\item \textbf{Law of Truly Large Numbers}: Rare events are expected given the volume of global human interactions.
\item \textbf{Cognitive Biases}: Confirmation bias and the availability heuristic skew perceptions of randomness.
\item \textbf{Random Walk and Network Theory}: Human mobility models and social graphs (small-world networks) explain recurring encounters.
\end{itemize}

While these theories robustly model most coincidences, they struggle to account for repeated, high-improbability re-encounters—especially those not explainable by social network proximity or mobility models.

\section{Empirical Pathways for Validation}

\subsection{Quantum Detection}
\begin{itemize}[nosep]
\item Use quantum sensors or imaging to detect non-classical signals during interpersonal interaction.
\item Investigate coherence patterns under physiological conditions.
\end{itemize}

\subsection{Statistical Deviation Detection}
\begin{itemize}[nosep]
\item Analyze large-scale human mobility datasets for anomalies diverging from known random walk distributions.
\item Examine clusters of anomalous re-encounters across social and physical distance.
\end{itemize}

\subsection{Neurobiological Correlation}
\begin{itemize}[nosep]
\item Use fMRI, EEG, or NMR to detect neurophysiological shifts in entanglement-laden events.
\item Map emotional resonance to potential Broctyte activation.
\end{itemize}

\section{Literature Review and Contextual Foundations}

See attached bibliography section for details.


\begin{figure}[htbp]
\centering
\includegraphics[width=0.85\textwidth]{General_Theory_Diagram.pdf}
\caption{Visual representation of the General Theory of Human Atomic Intertwining.}
\label{fig:atomic-intertwining}
\end{figure}

\section{Conclusion and Call for Collaboration}

This theory acknowledges its speculative nature but encourages serious investigation through interdisciplinary collaboration. If entanglement renewal can be empirically distinguished from stochastic explanations, this would redefine our understanding of consciousness, agency, and human connection.

\bibliographystyle{plainnat}
\bibliography{references}

\end{document}
